\chapter{آنتروپی و استفاده از آن در نهان~نگاری}\label{ch:Entropy}
 
\section{مقدمه}

در فصل گذشته سیستم بینایی انسان و ویژگی‌های آن را مورد بررسی قرار دادیم.
در حوزهٔ تبدیل \lr{DCT} و تبدیل موجک، دو مدل معروف و موجود برای سیستم بینایی
معرفی کردیم. همچنین  چارچوب کلی استفاده از مدلهای بینایی را برای  کاربرد
نهان نگاری، مطرح نمودیم و دو طرح مرجع \lr{P\&Z } و \lr{K\&R} را نیز مورد بررسی قرار دادیم. 
در این فصل به بیان هدف اصلی این پایان نامه که بررسی اثر آنتروپی در نهان~نگاری است، می‌پردازیم.
لذا ابتدا در بخش \ref{sec:entropy} به  بیان مفهوم آنتروپی پرداخته، سپس در 
بخش \dots 

\section{آنتروپی}\label{sec:entropy}

در این قسمت ابتدا توضیح مختصری در بارهٔ مفهوم آنتروپی داده خواهد شد. 

