\chapter{پیاده سازی و نتایج}

\section{پیاده سازی پلتفرم}
\subsection{سیستم مدیریت}

در پلتفرم‌های بزرگ، نیاز به سیستمی برای مدیریت پلتفرم به وضوح احساس می‌شود. این سیستم باید قابلیت هماهنگی و یکپارچگی بین اجزای مختلف را داشته باشد تا اطمینان حاصل شود که همه بخش‌ها به درستی و بدون مشکل عمل می‌کنند. ویژگی‌های حیاتی این سیستم شامل استفاده از ابزارهای خودکارسازی و نظارت پیشرفته، مدیریت بهینه منابع سخت افزاری، پیاده‌سازی فرآیندهای مستمر بهبود و به‌روزرسانی، مدیریت دسترسی‌ها و امنیت و مستندسازی فرآیندها و تغییرات است. سیستم مدیریت شامل تمامی ابزارهای مدیریتی مانند مدیریت مخازن مولفه‌ها، کد و خط لوله \lr{CI/CD}، مانیتورینگ کل سیستم و جمع آوری لاگ است. علاوه بر این، استراتژی استقرار پروژه، اعمال مهاجرت‌ها\footnote{\lr{Migrations}}، پیکربندی پروژه‌ها، مدیریت سرورهای \lr{DNS} و \lr{NTP}\footnote{\lr{Network Time Protocol}} و استفاده از ابزارهایی مانند \lr{Foreman} برای نصب سیستم‌عامل به صورت \lr{PXE} نیز توسط همین سیستم مدیریت می‌شود.

به منظور پیاده سازی این سیستم، ما دو ماشین مجزا برای مدیریت پلتفرم به منظور ایجاد قابلیت تحمل خطا\footnote{\lr{Fault Tolerance}} و دسترسی پذیری بالا\footnote{\lr{High Availability}}  قرار می‌دهیم. از آنجایی که فرآیند و پروسه سنگینی روی این ماشین ها انجام نمی شود مشخصات کمتری می تواند به نسبت ماشین های پروژه داشته باشد. مشخصات سخت افزاری هرکدام از این ماشین ها در جدول
~\ref{tb: management conf}
قرار دارد.  این ماشین ها به صورت ماشین مجازی با استفاده از \lr{OpenStack} ساخته می شود. ماشین‌های مدیریت باید برای نصب و راه‌اندازی ابزارهای مدیریت پلتفرم پیکربندی شوند و این کار با استفاده از ابزار \lr{Ansible} انجام می گیرد. به این منظور \lr{Role} های مشخصی برای هر بخش نوشته شده است تا بتوان بدون هیچ کار دستی و به صورت کاملا خودکار سیستم ها را پیکربندی کرد. این \lr{Role} های انسیبلی با استفاده از قاعده نسخه گذاری \lr{Semantic} نسخه گذاری شده و در مخزن \lr{raw} موجود در \lr{Nexus} که یک ابزار مدیریت مخازن مولفه می باشد نگه داری خواهند شد. در نهایت برای پیکربندی سیستم، \lr{Role} موردنظر با نسخه مشخص از \lr{Nexus} گرفته شده و بااستفاده از \lr{Ansible} ماشین ها پیکربندی می شوند. از آنجایی که این پیکربندی در محیط های مختلف مانند توسعه و عملیات می تواند متفاوت باشد، ما با استفاده از قابلیت \lr{Overriding} در این ابزار مقادیر پیش فرض را برای هر محیط تغییر خواهیم داد. به همین منظور اسکریپتی طراحی شده که در لینک 
گیت هاب\footnote{\url{https://github.com/abolfazlyarian/mlops.git}} قابل مشاهده است.

\begin{table}
	\centering
	\caption{مشخصات سخت افزاری ماشین های مدیریت}
	\label{tb: management conf}
	\begin{tabular}{|c|c|c|c|}
		\hline
		\lr{OS} & \lr{Storage} &  \lr{RAM} & \lr{CPU} \\ \hline
		\lr{Ubuntu 18.04} & \lr{512 GB} & \lr{8 GB} & \lr{4 Core} \\ \hline
	\end{tabular}
\end{table}

پروسه پیکربندی و نصب ابزار در ماشین های مدیریت به صورت زیر انجام می گردد:
\begin{enumerate}
	\item 
	پیکربندی ماشین ها:
این قسمت شامل نصب و پیکربندی ابزارهایی نظیر
\lr{BIND}
برای سرور \lr{DNS}،
\lr{APT}
برای مدیریت ابزار در سیستم عامل \lr{Ubuntu}،
\lr{pip}
برای مدیریت کتابخانه ها پایتون،
\lr{chrony}
برای سرور \lr{NTP}
،\lr{LDAP} 
برای مدیریت کاربران و … می باشد.
	\item
نصب و پیکربندی \lr{Docker}: از آنجایی که مدیریت ابزارها به صورت کانتینر مناسب تر است، برای انجام مراحل بعدی نیاز به نصب \lr{Docker} می باشد. پس از نصب به منظور ذخیره سازی تمام مولفه ها مورد استفاده به مخزن ساخته شده در \lr{Nexus} مدیریت متصل خواهد شد.
	\item 
 \lr{Nexus}: از این ابزار به منظور مدیریت مخازن مولفه ها استفاده شده است. مخازن مورد استفاده ما \lr{APT}، 
	\lr{pip}،
	\lr{Docker}
	و \lr{raw} می باشد (شکل 
	~\ref{fig: nexus repo}).

	\item 
 \lr{Jenkins}: از این ابزار به منظور اجرا و مدیریت خط لوله های \lr{CI/CD} پروژها و هم چنین پیکربندی آن ها توسط مدیران سیستم استفاده می شود (شکل 
	~\ref{fig: jenkins}).
	
	\item 
	نصب \lr{GitLab}: به منظور مدیریت کد در پروژه ها و هم چنین مدیریت \lr{Role}های انسیبلی برای پیکربندی پروژها استفاده می شود (شکل 
	~\ref{fig: gitlab}).
	
\begin{figure}[!t]
	\centering
	\includegraphics[scale=0.3]{nexus-repo.png}
	\caption{مخازن مولفه در \lr{Nexus}}
	\label{fig: nexus repo}
\end{figure}
\begin{figure}[!t]
	\centering
	\includegraphics[scale=0.3]{jenkins1.png}
	\caption{خط لوله ها \lr{CI/CD} در \lr{Jenkins}}
	\label{fig: jenkins}
\end{figure}
\begin{figure}[!t]
	\centering
	\includegraphics[scale=0.3]{gitlab.png}
	\caption{مخازن کد در \lr{Gitlab}}
	\label{fig: gitlab}
\end{figure}
\end{enumerate}
\clearpage
\subsection{خوشه کوبرنتیز}
در طراحی یک پلتفرم \lr{MLOps} جامع و کارآمد که تمامی ابزارهای مورد نیاز را در بر می‌گیرد، هدف اصلی ایجاد یک بستر یکپارچه، مقیاس‌پذیر و انعطاف‌پذیر برای مدیریت چرخه حیات مدل‌های یادگیری ماشین است. این پلتفرم شامل مجموعه‌ای از ابزارها و تکنولوژی‌های متن‌باز است که همگی روی خوشه کوبرنتیز مستقر می‌شوند. استفاده از کوبرنتیز در پلتفرم‌های \lr{MLOps} به دلیل قابلیت‌های منحصر به فرد آن در مدیریت خودکار، مقیاس‌پذیری و کارایی منابع است. 







