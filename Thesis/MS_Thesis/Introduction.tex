\chapter{مقدمه} \label{ch:Introduction}

گسترش روز افزون شبکه جهانی اینترنت و توسعه فناوری اطلاعات، نیاز فزاینده‌ای را به استفاده از
سرویسهای چندرسانه‌ای دیجیتال،  در پی داشته به طوریکه کاربردهای دیجیتال شاهد رشد
شگرفی در طول دهه گذشته بوده است که نتیجه  آن ایجاد سیستمهای  کارآمد در ذخیره، انتقال 
و بازیابی اطلاعات است. مزایای فراوان فناوری دیجیتال، باعث محبوبیت و کاربرد هر~
چه بیشتر آن توسط اشخاص شده تا جاییکه حتی وسایل ضبط و پخش صدا و تصویر آنالوگ خانگی
هم به سرعت با نمونه‌های دیجیتال جایگزین شده‌اند. اما این موضوع مسائل حاشیه‌ای دیگری
برای بشر ایجاد نموده است. به طوریکه امکان تهیه کپی‌های متعدد از روی نسخه اصلی 
بدون کاهش کیفیت آن و یا سادگی جعل و تغییر محتوای اطلاعاتی نسخه اصلی، باعث شده که 
مالکیت  معنوی\footnote{\lr{Intellectual Property}} صاحبان اثر به خطر افتاده و در 
نتیجه بسیاری از ارائه دهندگان سرویسهای چندرسانه‌ای (از جمله شرکتهای فیلم‌سازی)
از ارائه نمونه دیجیتال محصولاتشان خودداری نمایند. لذا برطرف نمودن این مشکلات،
یکی از زمینه‌های پژوهشی مهم در عرصه مخابرات و بخصوص پردازش سیگنال است. 

              
    
