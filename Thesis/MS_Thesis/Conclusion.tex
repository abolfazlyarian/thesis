\chapter{نتیجه‌گیری و پیشنهادات} \label{ch:Conclusion}

ما یک پلتفرم \lr{MLOps} متن‌باز و مبتنی بر ابر طراحی کرده‌ایم که خودکارسازی و بازتولید را در اکثر مراحل فرآیند یادگیری ماشین فراهم می‌کند. این پلتفرم در محیط‌های کوبرنتیز ارائه‌دهندگان مختلف ابری، کوبرنتیز داخلی و کوبرنتیز شبیه‌سازی شده روی ماشین‌های محلی اجرا می‌شود. هدف این طراحی، ایجاد زیرساختی است که مناسب اکثر پروژه‌ها و تیم‌های یادگیری ماشین باشد و فرآیندهای آن‌ها را به صورت خودکار و مقیاس‌پذیر نماید. 

محدودیت‌های فعلی و کارهای آینده شامل چندین بهبود مهم است که باید در پلتفرم \lr{MLOps} ما اعمال شود. در حال حاضر، فرآیندهای بازآموزی توسط یک وب‌هوک ساده تحریک می‌شوند که نیازمند یک اپراتور سفارشی برای مدیریت هوشمندانه‌تر چرخه‌ها و جلوگیری از بازآموزی‌های مکرر است. همچنین، پلتفرم برای اجرا به زیرساختی با منابع سخت‌افزاری زیاد نیاز دارد که توسعه محلی را دشوار و هزینه‌های سخت‌افزاری را افزایش می‌دهد؛ بنابراین، بهینه‌سازی‌های بیشتری برای سبک‌تر کردن پلتفرم ضروری است. بهینه‌سازی سرویس مش با جایگزینی \lr{Istio} با یک سرویس مش سبک‌تر، ادغام استفاده از سرویس مش با سرویس های کوبرنتیز به منظور افزایش سرعت پاسخ دهی به درخواست کاربر و کاهش اجزای غیرضروری \lr{Knative-Serving} نیز مورد نیاز است. امنیت سایبری پلتفرم باید تقویت شود و یک مطالعه موردی برای ارزیابی عملکرد و قابلیت اجرایی پلتفرم با پیاده‌سازی سیستم‌های یادگیری ماشین پیچیده و پرمنبع مانند آموزش مدل‌های زبانی بزرگ انجام شود. بهبودهای آینده شامل توسعه مرحله مدیریت شده برای برچسب‌گذاری داده‌ها، بهبودهای رابط کاربری برای ساده‌سازی استفاده از پلتفرم و ایجاد فرم‌های رابط کاربری برای افزودن استقرارها، جریان‌های کاری و تحریک‌ها است.
