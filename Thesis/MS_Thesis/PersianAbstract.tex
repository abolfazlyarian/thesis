\thispagestyle{empty}
\phantom{a}
\vfil
%\voffset=1cm

\begin{center}
\begin{minipage}{0.9\textwidth}


\noindent \textbf{چکیده:} 
\vskip 2mm \par

در دنیای دیجیتال امروزه، نهان~نگاری مقاومِ تصویر که در آن یک سیگنال حامل داده به صورت 
نامرئی و مقاوم در برابر حملات در تصویر تعبیه می‌شود، به عنوان یک راهکار برای حل 
مساله حفاظت از حق تالیف محصولات تصویری معرفی شده است. برای این منظور تاکنون جهت نهان~نگاری
روشهای متعددی به کار گرفته شده است که از آن جمله می‌توان به استفاده از مدلهای بینایی جهت 
یافتن میزان بیشینهٔ انرژی نهان~نگاره برای تعبیه در تصویر و استفاده از حوزه های مقاوم 
در برابر حملات، اشاره نمود. در همین راستا در این پایان~نامه به استفاده از مفاهیم حوزه تئوری 
اطلاعات به عنوان یک راهنما در توسعهٔ الگوریتمهای موجود، جهت قرار دادن بهینه نهان~نگاره پرداخته
شده است. همچنین در ساختار پیشنهادی که برای افزایش مقاومت در حوزه تبدیل تصویر پیاده می‌شود، 
از تبدیلات چنددقتی مانند تبدیل موجک گسسته و تبدیل \lr{MR-SVD} که به سیستم بینائی  انسان نزدیکترند،
استفاده می‌شود. به طوریکه در حوزه تبدیل موجک، با استفاده از آنتروپی و تاثیر پدیدهٔ پوشش 
آنتروپی به اصلاح مدلهای بینایی مرتبط با این حوزه پرداخته و بدین ترتیب نهان~نگاره با قدرت 
و مقاومت بالاتر در تصویر تعبیه نموده و همچنین کیفیت بهتر برای تصویر نهان~نگاری شده بدست آمد.
همچنین در حوزهٔ تبدیل \lr{MR-SVD} ابتدا این تبدیل که تاکنون برای نهان~نگاری استفاده نشده بود،
جهت نهان~نگاری بکار گرفته شد و سپس مشابه ساختار پیشنهادیِ مبتنی بر آنتروپی در حوزه تبدیل 
موجک، در حوزه این تبدیل  نیز بکار رفت و نتایج شبیه‌سازیها مقاوم‌تر بودن ساختار پیشنهادی 
و کیفیت بالاتر تصویر نهان~نگاری شده در این حوزه را نتیجه داد.

\vspace{15mm}
\noindent \textbf{کلمات کلیدی:}
\vskip 2mm
\begin{tabular}{rr}
۱- نهان~نگاری تصویر & \lr{\,Image Watermarking}. \\
۲- تبدیل چنددقتی &  \lr{\,Multi-Resolution Transform}. \\
۳- سیستم بینایی انسان & \lr{\,Human Visual System (HVS)}. \\
۴- تبدیل موجک & \lr{\,Wavelet Transform}. \\
۵- تجزیه مقادیر تکین & \lr{\,Singular Value Decomposition (SVD)}. \\
۶- آنتروپی & \lr{\,Entropy}. \\
۷- پوشش آنتروپی & \lr{\,Entropy Masking}. 

\end{tabular}

%\noindent \rule{\textwidth}{2pt}


\end{minipage}
\end{center}
\vfil

