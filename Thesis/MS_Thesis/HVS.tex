\chapter{\lr{MLOps}}\label{ch:HVS}
 
\section{مقدمه}
 
در حالی که مدل‌های یادگیری ماشین به طور گسترده‌ توسعه یافته‌اند، انتقال آن‌ها از مفهوم آزمایشی به محیط تولید اغلب با شکست مواجه می شود. این فاصله بیشتر به خاطر این است که تاکنون توجه اصلی روی ساخت مدل‌ها بوده است، نه روی تولید محصولات یادگیری ماشین که قابلیت استفاده در محیط تولید را دارند. علاوه بر آن، مدیریت بخش‌ها و زیرساخت‌های پیچیده‌ای که برای یک استقرار موثر ضروری هستند نیز در این امر مغفول مانده اند. برای رفع این مسئله، مفهوم عملیات یادگیری ماشین یا \lr{MLOps} معرفی شده است. \lr{MLOps} بر روی خودکارسازی و عملیاتی کردن فرآیندهای یادگیری ماشین تمرکز دارد تا انتقال پروژه‌های یادگیری ماشین از مفهوم به تولید را تسهیل کند. این رویکرد شامل دیدگاه جامعی از طراحی سیستم، هماهنگی اجزا، تعریف نقش‌ها و مسئولیت‌ها می باشد. هدف کاهش خطا به منظور افزایش قابلیت اطمینان و کارایی سیستم‌های یادگیری ماشین در کاربردهای واقعی می باشد. این فصل به بررسی تعریف، اصول، ابزار و معماری جامعی از یک پلتفرم  \lr{MLOps} پرداخته و در نهایت، محصولات و رقبا این حوزه را بررسی می کنیم.

\section{تعریف مفاهیم اولیه}

\lr{MLOps}
 یا عملیات یادگیری ماشین به مجموعه‌ای از فرایندها، ابزارها و شیوه‌ها جهت مدیریت چرخه توسعه مدل‌های یادگیری ماشین در یک محیط عملیاتی اشاره دارد. همچنین این چرخه شامل همکاری بین دانشمندان داده و مهندسان \lr{DevOps} است به گونه ای که این اطمینان حاصل شود که مدل‌ها به طور مؤثر توسعه، استقرار، پایش و به‌روزرسانی می‌شوند. هدف \lr{MLOps} افزایش سرعت، قابلیت اطمینان و مقیاس پذیری مدل‌های یادگیری ماشین و فرایند توسعه این مدل ها در تولید است؛ درحالی‌که خطرات ناشی از ریسک عدم موفقیت را نیز کاهش می‌دهد. همچنین به‌کارگیری \lr{MLOps} فرایند مدیریت را ساده‌تر کرده، کیفیت را افزایش می‌دهد و استقرار مدل‌های یادگیری عمیق و یادگیری ماشین در محیط‌های تولید با مقیاس بزرگ را خودکار می‌کند. لذا می توان گفت یکی از اهداف \lr{MLOps}، بهبود خودکارسازی و ارتقای کیفیت مدل‌های تولید و درعین‌حال توجه به الزامات تجاری و نظارتی است. 
 
 
 استقرار مدل‌های یادگیری ماشین روی محیط عملیاتی در \lr{MLOps} اهمیت زیادی دارد، زیرا به سازمان‌ها کمک می‌کند تا مطمئن شوند که مدل‌هایشان در طول زمان دقیق، قابل‌اعتماد و کارآمد هستند. به‌طورکلی، \lr{MLOps} با خودکار کردن بسیاری از مراحل مربوط به استقرار و مدیریت مدل‌های یادگیری ماشین، به دانشمندان و مهندسان داده اجازه می‌دهد تا با همکاری یکدیگر به ارائه سریع‌تر و کارآمدتر مدل‌های یادگیری ماشین دست یابند. 
\subsection{اصول}
 برای تسهیل در رسیدن به اهداف فوق، تیم‌های \lr{MLOps} از اصول زیر استفاده می کنند:
\begin{enumerate}
	\item 
	خط لوله خودکار \lr{CI/CD} و هماهنگ سازی جریان کاری:
	خودکارسازی \lr{CI/CD} شامل مراحل ساخت، آزمایش، تحویل و استقرار است که به توسعه‌دهندگان نسبت به موفقیت یا شکست مراحل مختلف بازخورد سریعی را ارائه داده و بهره‌وری کلی را افزایش می‌دهد \cite{MLOpsPipeline1}. در همین حال، هماهنگ سازی جریان کاری وظایف یک خط لوله‌ یادگیری ماشین را با استفاده از گراف‌های بدون‌حلقه‌ی جهت‌دار\footnote{\lr{Directed Acyclic Graph (DAG)}} هماهنگ می‌کند، که ترتیب اجرای وظایف را با توجه به روابط و وابستگی‌ها تعیین می‌کند. ترکیب این دو رویکرد می ‌تواند به بهبود عملکرد و کارایی تیم‌های توسعه و داده‌کاوی کمک کند \cite{MLOpsWO1}.
	\item 
	کنترل نسخه مدل‌های یادگیری ماشین، مجموعه‌داده‌ها و کد منبع:
با استفاده از نسخه‌بندی مدل، داده و کدمنبع، می‌توان هر تغییر و اصلاحی را در طول زمان دنبال کرد، که این امر به توسعه‌دهندگان و محققان اجازه می‌دهد تا به راحتی به نسخه‌های قبلی بازگردند و نتایج را بازبینی کنند. این قابلیت برای حفظ یکپارچگی و شفافیت در پروژه‌های نرم‌افزاری و علمی بسیار حیاتی است \cite{MLOpsPipeline1}.
	\item 
	آموزش و ارزیابی مدوام مدل یادگیری ماشین:
آموزش مداوم\footnote{\lr{Continuous Training (CT)}} در یادگیری ماشین به معنای آموزش دوره‌ای مدل‌های یادگیری ماشین بر اساس داده‌های جدید است. این فرآیند همیشه شامل یک مرحله ارزیابی برای سنجش تغییرات کیفیت مدل است. یکی از دلایل لزوم آموزش مداوم، رانش داده یا مدل\footnote{\lr{Data or Model Drift}} است، که به تغییرات تدریجی در داده‌ها یا عملکرد مدل در طول زمان اشاره دارد و می‌تواند باعث کاهش دقت پیش ‌بینی‌ها شود. در \lr{MLOps}، آموزش مداوم به توسعه‌دهندگان کمک می‌کند تا مدل‌ها را به‌روز نگه دارند و به تغییرات محیطی یا زمانی واکنش نشان دهند \cite{MLOpsCT1,MLOpsCT2}.
	\item 
	نظارت مستمر و ثبت گزارش برای ردیابی عملکرد مدل
	\item 
	حلقه های بازخورد\footnote{\lr{feedback loops}}
\end{enumerate}



